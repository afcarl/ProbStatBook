\chapter{Probability spaces}

\section{Definition}

In order to properly understand a statement like
\begin{quote}
``the chance of getting a flush in five-card poker is about 0.2\%
(flush = all five cards are from the same suit)'',
\end{quote}
we need to specify the underlying {\it probability space}. This has two components:
\begin{enumerate}
\item A {\it sample space} or {\it space of outcomes}.

This is $\Omega = \{\mbox{all possible five-card hands}\}$.
\item The {\it probabilities of outcomes} Each outcome is assigned a
  probability, which is a non-negative real number, such that the sum
  of the probabilities over all of the outcomes is 1:
$$ \sum_{\omega \in \Omega} \pr(\omega) = 1 .$$
\end{enumerate}

In our example, assuming the cards are dealt fairly, all outcomes are equally probable, so
$$\pr(\omega) = 1/|\Omega| \mbox{\ \ \ \ \ for all $\omega \in \Omega$.} $$.

{\bf Events} are subsets of $\Omega$, in our example, the event of
interest is $A = \{\omega: \mbox{$\omega$ is a flush}\}$. This is a
subset of $\Omega$; that is, $A \subset \Omega$. Pictorially we can
represent the situation thus:

\begin{center}
%%\resizebox{1.5in}{!}{\input{figs/1.1.pstex_t}}
\end{center}

\noindent
The outer box is $\Omega$; every point in it is a particular five-card hand $\omega$. The inner set is $A$, and the probability that it occurs is
$$ \pr(A) = \sum_{\omega \in A} \pr(\omega) .$$
In general, if $\Omega$ is finite, the probability of events is the
sum of the probabilities of the outcomes in that event.\footnote{If
  $\Omega$ is continuous then this relationship does not hold and we
  need to define event probabilities differently. We will get to this
  a bit later. Till then we will only discuss finite outcome spaces,
  also referred to as ``discrete probability spaces''.}

In our example, since all outcomes are equally likely, $\pr(A) =
|A|/|\Omega|$. We now calculate this ratio.

The size of $\Omega$ is the number of 5 card hands is 
$$ |\Omega| = {52 \choose 5} = 2,598,960$$
The number of hands that are flush is the
number of suits (4) times the number of hands that can be chosen from
a single suit: 
$$|A|= 4\times {13 \choose 5} = 5,148$$ 
Thus the probability of a flush is
$$\frac{5,148}{2,598,960} \approx 0.00198$$
Which, as promised, is approximately 0.2\%.

\section{A first set of canonical examples}

\begin{enumerate}
\item {\it Roll a die.} What is the chance of getting a number $> 3$?

Probability space: sample space $\Omega = \{1,2,3,4,5,6\}$; probabilities $\pr(\omega) = 1/6$.

Event of interest: $A = \{4,5,6\}$; $\pr(A) = 1/2$.

\item {\it Roll three dice.} What is the chance their sum is 3?

The sample space is 
\begin{eqnarray*}
\Omega 
& = & \{(1,1,1), (1,1,2), \ldots, (6,6,6)\} \\
& = & \{(d_1, d_2, d_3): 1 \leq d_1, d_2, d_3 \leq 6\} \\
& = & \Omega_o \times \Omega_o \times \Omega_o 
\ \ \ = \ \ \ \Omega_o^3 \mbox{\ \ \ \ where\ } \Omega_o = \{1,2,3,4,5,6\}.
\end{eqnarray*}
The probabilities of outcomes $\omega \in \Omega$ are $\pr(\omega) = 1/6^3 = 1/216$.

The event of interest is $A = \{(1,1,1)\}$ whose probability is $\pr(A) = 1/216$.

\item {\it Roll $n$ dice.}

Sample space $\Omega = \Omega_o^n$, where $\Omega_o = \{1,2,3,4,5,6\}$. Each outcome $\omega \in \Omega$ has probability $1/|\Omega| = 1/6^n$.

\item {\it Socks in a drawer.} A drawer contains three blue socks and three red socks. You put your hand in and pick out a random sock. Then you put your hand in again and pick out another random sock. What's the chance the two of them match?

There are several ways to set up the sample space, but one possibility is to have a tuple whose first coordinate is the color of the first sock and whose second coordinate is the color of the second sock. $\Omega = \{B,R\} \times \{B,R\}$. The probabilities of outcomes are
\begin{eqnarray*}
\pr((B,B)) & = & \frac{3}{6} \cdot \frac{2}{5} \ \ = \ \ \frac{1}{5} \\
\pr((B,R)) & = & \frac{3}{6} \cdot \frac{3}{5} \ \ = \ \ \frac{3}{10} \\
\pr((R,B)) & = & \frac{3}{6} \cdot \frac{3}{5} \ \ = \ \ \frac{3}{10} \\
\pr((R,R)) & = & \frac{3}{6} \cdot \frac{2}{5} \ \ = \ \ \frac{1}{5}
\end{eqnarray*}
(Notice that they add up to 1.)

The event of interest is $A = \{(B,B), (R,R)\}$, which has probability $2/5$.

\item {\it Socks in a drawer, again.} This time the drawer has three blue socks and four red socks.

The sample space $\Omega$ is the same, but the probabilities of outcomes are different:
\begin{eqnarray*}
\pr((B,B)) & = & \frac{3}{7} \cdot \frac{2}{6} \ \ = \ \ \frac{1}{7} \\
\pr((B,R)) & = & \frac{3}{7} \cdot \frac{4}{6} \ \ = \ \ \frac{2}{7} \\
\pr((R,B)) & = & \frac{4}{7} \cdot \frac{3}{6} \ \ = \ \ \frac{2}{7} \\
\pr((R,R)) & = & \frac{4}{7} \cdot \frac{3}{6} \ \ = \ \ \frac{2}{7}
\end{eqnarray*}
The event of interest is $A = \{(B,B), (R,R)\}$, which has probability $3/7$.

\item {\it Shuffling a deck of cards.} You randomly shuffle a deck of $52$ cards and lay them out before you.

Here $\Omega = \{\mbox{all possible orderings of 52 cards}\}$. One way to compute $|\Omega|$ is to reason that there are 52 choices for what the first card in the sequence will be, 51 choices for the second card, 50 choices for the third card, and so on. Therefore
$$ |\Omega| = 52 \cdot 51 \cdot 50 \cdots 2 \cdot 1 .$$
This expression is called 52! (``52 factorial''). It is the number of {\it permutations} of 52 elements.

\end{enumerate}

\section{Tossing a fair coin}
\label{sec:FairCoin}

Suppose you toss a fair coin. The sample space is $\Omega_o = \{H,T\}$ (heads or tails), and each of the two outcomes has probability exactly $1/2$. What is more interesting is to toss the coin multiple times, independently. 

\subsection{Toss a fair coin 10 times}

Now the sample space is $\Omega = \Omega_o^{10}$; it includes, for instance, the sequence $(H,T,H,T,H,T,H,T,H,T)$. Since $|\Omega| = 2^{10} = 1024$, each element in $\Omega$ has probability exactly $1/1024$.

\begin{enumerate} 

\item What is the chance that {\it none} of the coin tosses are heads?

The event of interest is $\{(T,T,T,T,T,T,T,T,T,T)\}$, whose probability is $1/1024$.

\item What is the chance of exactly {\it one} head?

Now the event is 
$$A = \{(H,T,T,T,T,T,T,T,T,T), (T,H,T,T,T,T,T,T,T,T), \ldots, (T,T,T,T,T,T,T,T,T,H)\} .$$
Each sequence in $A$ is completely determined by the location of the $H$ within it. There are 10 possible locations; therefore $|A| = 10$, whereupon $\pr(A) = |A|/|\Omega| = 10/1024$.

\item What is the chance of exactly {\it nine} heads?

Equivalently, what is the chance of exactly one tail? This is the same calculation as before, $10/1024$.

\item What is the chance of exactly {\it two} heads?

The sequences of interest are $A = \{\omega \in \Omega: \mbox{$\omega$ has exactly 2 heads}\}$. Each such sequence is specified by the locations of its two heads; write these as a pair $(i,j)$, where $1 \leq i,j \leq 10$ and $i \neq j$. For instance, $(7,5)$ refers to $(T,T,T,T,H,T,H,T,T,T)$. 

The number of such pairs is $10 \cdot 9 = 90$ (10 choices for $i$, and thereafter just 9 choices for $j$). But this ends up double-counting sequences, because for instance, the pair $(5,7)$ also refers to $(T,T,T,T,H,T,H,T,T,T)$. Each sequence in $A$ is counted twice -- it corresponds to two pairs -- and therefore $|A| = 45$.

Finally, $\pr(A) = |A|/|\Omega| = 45/1024$.

\item What is the chance of exactly {\it four} heads?

This time, any sequence in $A = \{\omega: \mbox{has four heads}\}$ can be written as a 4-tuple $(i,j,k,l)$, where $1 \leq i,j,k,l \leq 10$ and $i \neq j \neq k \neq l$. For instance,
$(7,2,4,9)$ denotes $(T,H,T,H,T,T,H,T,H,T)$. The number of such 4-tuples is $10 \cdot 9 \cdot 8 \cdot 7 = 10!/6!$.

But once again, there is overcounting. $(7,2,4,9)$ refers to the same sequence as $(2,4,7,9)$ and $(2,7,9,4)$ and $(9,7,2,4)$ and many other 4-tuples. How many of them? The number of permutations of the four elements $2,4,7,9$, namely $4!$.

Therefore
$$ |A| = \frac{10!}{4!6!} = 210$$
and $\pr(A) = 210/1024$.

\item What is the chance of exactly {\it six} heads?

This is the same as the chance of four tails, which is identical to the previous calculation.

\end{enumerate}

In the last few calculations, we had to find the number of ways of choosing $k$ positions out of $n$ available slots (for instance, choosing 4 positions out of 10 slots where a $H$ might occur). Generalizing the argument above, the number of ways to do this is
$$ \frac{n!}{k!(n-k)!}, \mbox{\ which we write as\ } {n \choose k} $$
(pronounced ``$n$ choose $k$''). This is also called a {\it binomial coefficient}.

\subsection{Toss a fair coin $n$ times}

Now the sample space is $\Omega = \{H,T\}^n$, with each sequence of $n$ outcomes having probability exactly $1/2^n$. Let $A_k$ denote the event that the sequence has $k$ heads.

Notice that the events $A_0, A_1, \ldots, A_n$ are {\it disjoint} (if $A_i$ occurs then $A_j$ cannot occur for $j \neq i$). Moreover,
$$ \Omega = A_0 \cup A_1 \cup \cdots \cup A_n .$$
Therefore,
$$ \sum_{k=0}^n \pr(A_k) \ \ = \ \ \pr(A_0) + \pr(A_1) + \cdots + \pr(A_n) \ \ = \ \ 1 ,$$
or equivalently,
$$ \sum_{k=0}^n |A_k| \ \ = \ \ |A_0| + |A_1| + \cdots + |A_n| \ \ = \ \ |\Omega| .$$

In general, $|A_k|$ is the number of ways of placing $k$ heads in a sequence of size $n$; we've seen that this is ${n \choose k}$. And since $|\Omega| = 2^n$, the last equality tells us that
$$ \sum_{k=0}^n {n \choose k} \ \ = \ \ {n \choose 0} + {n \choose 1} + {n \choose 2} + \cdots + {n \choose n} \ \ = \ \ 2^n.$$

For example, take $n = 5$. Then
$$
|A_0| = {5 \choose 0} = 1, \  
|A_1| = {5 \choose 1} = 5, \ 
|A_2| = {5 \choose 2} = 10, \ 
|A_3| = {5 \choose 3} = 10, \ 
|A_4| = {5 \choose 4} = 5, \ 
|A_5| = {5 \choose 5} = 1
$$
and these add up to $2^5 = 32$.


\section{A second set of canonical examples}

\begin{enumerate}
\item {\it Rooks on a chessboard.} You place 8 rooks at random on a chessboard. What is the chance that they are non-attacking (that is, no rook is attacking another)?

To describe the sample space, number the squares in the chessboard as 1 through 64, and let the configuration of 8 rooks be given by a {\it set} of eight positions, $\omega \subset \{1,2,\ldots,64\}$. Thus:
$$\Omega = \{\omega \subset \{1,2,\ldots,64\}: |\omega| = 8 \} .$$
You should check that 
$$ |\Omega| \ \ = \ \ {64 \choose 8} $$
and that of these, only 8! configurations are non-attacking. 

Therefore
$$ \pr(\mbox{non-attacking configuration}) \ \ = \ \ \frac{8!8!56!}{64!} .$$

\item {\it Birthday paradox.} A room contains $n$ people. What is the chance that two of them have the same birthday?

The probability space is not properly specified, so we need to make some assumptions. First, we'll assume that the $n$ birthdays are independent (that is, a person's birthday is not influenced by anyone else's birthday). Second, we'll assume that all days are equally likely -- that is, the chance of a birthday falling on any particular day is exactly $1/365$ (we're also ignoring the issue of leap years).

Number the people $1,2,\ldots, n$, and number the days of the year $1,2,\ldots, 365$. We will represent the birthdays of the people in the room by an $n$-tuple $(\omega_1, \ldots, \omega_n)$, where $\omega_i \in \{1,2,\ldots,365\}$ is the birthday of the $i$th person. Thus $\Omega = \{1,2,\ldots, 365\}^n$ and each $\omega \in \Omega$ has probability exactly $1/365^n$.

The event of interest is
$$ A  \ \ = \ \ \{\omega: \mbox{$\omega_i = \omega_j$ for some $i \neq j$} \}.$$
This is a typical situation in which it is easier to analyze the {\it complement} of $A$ than $A$ itself (that is, it is easier to compute the probability that $A$ {\it doesn't} occur than the probability that it occurs).
$$ A^c \ \ = \ \ \Omega - A \ \ = \ \ \{\omega: \omega_1 \neq \omega_2 \neq \cdots \neq \omega_n\}.$$
In other words, $A^c$ is the event that everyone's birthday is different. What is the size of $A^c$? There are 365 choices for $\omega_1$, 364 for $\omega_2$, and so on, whereupon
$$ |A^c| \ \ = \ \ 365 \cdot 364 \cdot 363 \cdots (365-n+1) \ \ = \ \ \frac{365!}{(365-n)!}.$$Therefore
$$ \pr(A) \ \ = \ \ 1 - \pr(A^c) \ \ = \ \ 1 - \frac{365!}{(365-n)!365^n} .$$
This is exactly correct, but it is a little hard to understand intuitively. So let's do the calculation a different way, using an approximation.

A very useful fact is that for small $x$ (positive or negative), $e^x \approx 1+x$. And in fact, $e^x \geq 1+x$ no matter what $x$ is. Now let's return to the event $A^c$.
\begin{eqnarray*}
\pr(A^c) & = & \pr(\omega_2 \neq \omega_1) \cdot \pr(\omega_3 \neq \omega_1, \omega_2) \cdots \pr(\omega_n \neq \omega_1, \ldots, \omega_{n-1}) \\
& = & \left(1 - \frac{1}{365} \right) \left(1 - \frac{2}{365} \right) \cdots \left(1 - \frac{n-1}{365} \right) \\
& \leq & \exp(-1/365) \cdot \exp(-2/365) \cdots \exp(-(n-1)/365) \mbox{\ \ \ where $\exp(x)$ means $e^x$} \\
& = & \exp \left( - \frac{1}{365} \left(1 + 2 + \cdots + (n-1) \right) \right) \\
& = & \exp \left( - \frac{n(n-1)}{730} \right). 
\end{eqnarray*}
This upper bound is a very good approximation when $n$ is much smaller than $365$. 

Interestingly, when $n = 23$, we find that $\pr(A^c) \leq 0.5$, so $\pr(A) \geq 0.5$. That is, if there are 23 people in the room, chances are that two of them have the same birthday!

\item {\it Balls in bins.} You have $m$ indistinguishable balls and in front of you is a row of $n$ bins. You place each ball into a bin chosen at random.

Let's write the sample space as $\Omega = \{1,2,\ldots, m\}^n$; in each outcome $\omega = (\omega_1, \ldots, \omega_n)$, the value $\omega_i$ represents the number of balls in the $i$th bin.

Here are some interesting tidbits to prove.
\begin{itemize}
\item The chance that any particular bin is empty is at most $e^{-m/n}$.
\item If $m = 2n \ln n$, the chance that there exists an empty bin is at most $1/n$. To show this, it helps to use the {\it union bound}: for any events $A_1, \ldots, A_k$,
$$ \pr(A_1 \cup A_2 \cup \cdots \cup A_k) \ \ \leq \ \ \pr(A_1) + \pr(A_2) + \cdots + \pr(A_k).$$
\item The chance that no bin has 2 (or more) balls is at most $\exp(-m(m-1)/n)$. How is this related to the birthday paradox?
\end{itemize}
A lot of different probability spaces are simple cases of balls and bins. For instance, tossing a fair coin $m$ times is like throwing $m$ balls into $n=2$ bins (call one bin $H$ and the other bin $T$).

\section{Poker}

Poker is a family of card games wherein players make bets based on the
values of combinations of cards they possess or are shared among the
players. The eventual winner is determined by the player who survives
the betting and can form a hand with the highest value. Since making
bets requires an understanding of the potential value of a player's
hand, probability concepts can be used to make informed bets and
attempt to profit in the long run. Texas Hold'em is one of the most
popular variants of poker, and usually is the variant meant when
people say poker without specifying a particular game.

\subsection{Rules of Texas Hold'em}
In Texas Hold'em, there are five shared (or \emph{community} cards) and
two private (also known as \emph{hole} or \emph{pocket}) cards per
player. A hand consists of 5 cards in total, so at the end of the game
the value of a player's hand is determined by the highest possible
combination the player can make from the two \emph{hole} cards and the
\emph{community} cards. Each player is first dealt two cards, face
down. These cards are kept hidden until the end of the game.

Players then make bets in a clockwise circle, based on the potential
perceived value of their hand at the end of the game. When making
bets, players must \emph{call} in order to continue playing: they can
either \emph{check} the current bet, which means they must match the
highest bet previously made, or they can \emph{raise}, increasing the
value of the bet. If a player feels that their hand is not strong
enough to win, they can also \emph{fold}, or exit the game, but they
forfeit any money that they had previously bet. Bets contribute to the
\emph{pot}, the shared pool of money that the eventual winner will
claim. A round of betting continues until all players have either
\emph{checked} or \emph{folded}, i.e. the value of the bet is not
increased for one full rotation of the circle of players.

Here is a brief example of a round of betting, with hypothetical
players Alice, Bob, Carol, and Dan:
\begin{table}[h]
  \centering
  \begin{tabular}{ l || c | c | c}
    \hline
    Player & Bet & Terminology & Total value of Pot \\
    \hline
    Alice & \$2 &  & \$2 \\
    Bob & \$2 & Check & \$4 \\
    Carol & \$4 & Raise & \$8 \\
    Dan & \$4 & Check & \$12 \\
    Alice & \$4 & Check & \$14 \\
    Bob & \$2 & Fold & \$14 \\
    Carol & \$4 & Check & \$14 \\
  \end{tabular}
\end{table}
After Carol checks, everyone has either checked or folded and the
betting is over.

The initial betting is based only on the knowledge of the players' two
\emph{hole} cards. Then, the dealer deals the \emph{flop}, which are
the first three \emph{community} cards. Again, another round of
betting proceeds, with players now considering the best possible
combinations of their two \emph{hole} cards and the three cards in the
\emph{flop}, as well as the potential combinations achievable by the
last two unknown cards, and the possible combinations their opponents
may have. After the second round of betting, the dealer deals the
fourth community card, known as the \emph{turn}. Another round of
betting occurs, this time with players having knowledge of one more
card. After the third round of betting, the final community card, the
\emph{river}, is dealt. A fourth and final round of betting then takes
place, with the remaining players trying to estimate what potential
value their opponents may possess. Finally, if more than one player is
left, the \emph{showdown} occurs: players reveal the best five card
hand they can make from their \emph{hole} cards and the five
\emph{community} cards. The winner is the player who can form the
highest value hand, and takes the entire \emph{pot} as his
prize. While one cannot be sure of winning an individual game, the
goal of an intelligent poker player is to make a net profit over the
course of many games, by using the laws of probability (and
psychology) to ones advantage.

\subsection{Poker Hands}
In Texas Hold'em, hands consist of five cards. They fall under several
categories, with each category in the following list being higher than
all hands of lower categories. Within a category, hands are ordered by
the individual ranks of the cards, so for example a \emph{Flush} whose
highest card is the $10\clubsuit$ is higher ranked than a \emph{Flush}
whose highest card is the $9\heartsuit$, which is in turn higher than
any \emph{Straight}. Any leftover or unmatched cards (in the five card
hand) may be used to break ties.

\begin{enumerate}
\item Straight flush - five cards in sequence, all of the
  same suit, such as $Q\spadesuit J\spadesuit 10\spadesuit 9\spadesuit
  8\spadesuit$

\item Four of a kind - four cards of the same rank, eg. $8\spadesuit
  8\clubsuit 8\heartsuit 8\diamondsuite K\clubsuit$. In this case the
  $K\clubsuit$ would break ties.

\item Full house - three cards of one rank and two cards of another,
  eg. $6\clubsuit 6\heartsuit 6\spadesuit 9\heartsuit 9\clubsuit$.
\item Flush - five cards of the same suit, eg. $A\spadesuit
  J\spadesuit 10\spadesuit 6\spadesuit 4\spadesuit$.
\item Straight - five cards in sequence, not necessarily of the same
  suit. For example, 
\item Three of a kind - Three cards of the same rank, plus two other
  cards. For example, $10\spadesuit 10\clubsuit 10\heartsuit
  6\diamondsuit 4\diamondsuit$
\item Two pair - two pairs of cards of the same rank, plus a fifth
  card. For example, $J\clubsuit J\diamondsuit 8\spadesuit 8\clubsuit 5\heartsuit$
\item One pair - one pair of cards of the same rank, plus three
  others. For example, $K\clubsuit K\heartsuit J\spadesuit
  9\diamondsuit 4\diamondsuit$
\item High card - any hand not matching any of the above rules
  (i.e. no pairs, straights, or flushes). The value of the hand is
  determined by the value of the highest card. For example,
  $K\spadesuit J\heartsuit 8\clubsuit 7\clubsuit 4\spadesuit$

\end{enumerate}

You are dealt five cards at random from a deck of 52 cards. What is the chance of a flush? Of a straight flush? Of exactly one pair?

Let $\Omega = \{\mbox{all possible 5-card hands}\}$. Then $|\Omega| = {52 \choose 5}$ and each $\omega \in \Omega$ occurs with probability $1/|\Omega|$. 

Define three events of interest: 
\begin{eqnarray*}
F & = & \mbox{flush (all five cards of the same suit)} \\
S & = & \mbox{straight flush (same suit and consecutive)} \\
P & = & \mbox{the cards contain a single pair (eg. two 7s)}
\end{eqnarray*}

Then $|F| = 4 \cdot {13 \choose 5}$ (first choose a suit, then pick 5 cards from that suit), $|S| = 4 \cdot 10$ (first choose a suit, then choose the starting card in the sequence), and $|P| = 13 \cdot {4 \choose 2} \cdot 4^3 \cdot {12 \choose 3}$ (first choose which card occurs in the pair, then choose the two suits for that pair, then choose the suits of the remaining three cards, then choose their values).
 
\end{enumerate}

